\begin{frame}{Symplectic manifolds}
   \begin{definition}[Symplectic manifold] A \alert{symplectic manifold} is a pair $(M, \omega)$, where $M$ is a $2n$-dimensional manifold, and $\omega \in \Omega^2(M)$ is a closed, non-degenerate, $2$-form.
   \end{definition}
   \pause
   Thus, for every symplectic manifold we have an isomorphism induced by contraction $$TM \xrightarrow{\flat} T^\ast M; \,\, v \mapsto \iota_v \omega.$$ 
   \pause
   \begin{definition} For a subspace $i: W \hookrightarrow T_x M,$ define the \alert{symplectic orthogonal} as $$W^\perp := \{v \in T_q M, \,\, \omega(v, w) = 0, \forall w \in W \} = \ker i^\ast \circ \flat.$$
   \end{definition}

   $$\alert{\dim W^\perp = 2n - \dim W}$$
\end{frame}

\begin{frame}{Symplectic orthogonal}
\begin{definition} A subspace $W \subseteq T_xM$ (res. submanifold $L$) is called
\begin{itemize}
    \item \alert{isotropic} if $W \subseteq W^\perp$ (res. $T_x L \subseteq (T_x L)^\perp, \forall x \in L$);
    \item \alert{Lagrangian} if $W = W^\perp$ (res. $(T_x L)^\perp = T_x L, \forall x \in L$).
    \item \alert{coisotropic} if $ W^\perp \subseteq W^\perp$ (res. $(T_x L)^\perp \subseteq T_x L, \forall x \in L$).
\end{itemize}
\end{definition}
\pause
A isotropic submanifold is necessarily $n$-dimensional and we have the following characterization:
\begin{proposition} An $n$-dimensional submanifold $i: N \hookrightarrow M$ is Lagrangian if and only if $i^\ast \omega = 0.$
\end{proposition}
\end{frame}

\begin{frame}{Dynamics $=$ Lagrangian submanifolds}
    \begin{definition} Given a function $H \in C^\infty(M)$ define the \alert{Hamiltonian vector field} $X_H \in \mathfrak{X}(M)$ as the unique vector field sastifying $$\iota_{X_H}\omega = dH.$$ A vector field $X \in \mathfrak{X}(M)$ is called \alert{locally Hamiltonian} if $\iota_X \omega$ is closed.
\end{definition}
\pause
With the isomorphism $\flat: TM \rightarrow T^\ast M$ we can define $$\widetilde \omega := \flat ^\ast \omega_M.$$
\begin{theorem} A vector field $X: M \rightarrow TM$ is locally Hamiltonian if and only if it defines a Lagrangian submanifold.
\end{theorem}
\begin{proof} $X(M)$ is Lagrangian if and only if $$0 = X^\ast \widetilde \omega = -d \iota_X \omega.$$
\end{proof}
\end{frame}

\begin{frame}{Coisotropic reduction}
Given a coisotropic submanifold $i:N \hookrightarrow M$, the distribution $$x \mapsto (T_x N)^\perp$$ is regular and involutive. Therefore, it arises from a maximal foliation $\mathcal{F}$.
\pause
\begin{theorem} If $N/\mathcal{F}$ admits a smooth manifold structure such that $\pi: N \rightarrow N/\mathcal{F}$ defines a submersion, then there is an unique symplectic form $\omega_N$ on $N/\mathcal{F}$ such that $$\pi^\ast \omega_N = i^\ast \omega.$$ \alert{Furthermore, if $L$ is a Lagrangian submanifold in $M$, $\pi(L \cap N)$ is a Lagrangian submanifold in $N/\mathcal{F}.$} 
\end{theorem}

Allows for reduction of dynamics!
\end{frame}

\begin{frame}{Coisotropic reduction}
    \begin{proof} We omit the first part. For the projection of Lagrangian submanifolds, let $L$ be a Lagrangian submanifold and denote $L_N := \pi(L \cap N).$\\
    \vspace{0.5cm}

It is
sufficient to see that $\pi(L \cap N)$ is isotropic and that it has maximal dimension in $N / \mathcal{F}$. \pause It is isotropic since $[u] \in
T_q(L _ N)$ implies $\omega_N([u],[v]) = \omega(u,v) = 0$, for every $[v] \in T_q(L _ N)$.\pause  Now, since $\ker d_q\pi =
(T_qN)^{\perp_\omega}$, the kernel-range formula yields \begin{equation}\label{eq1symplectic} \dim L_N = \dim(L \cap N) - \dim (T_q L \cap (T_qN)^{\perp_\omega}). \end{equation} 
Furthermore, \begin{equation}\label{eq2symplectic} \dim(L \cap N) + \dim (T_qL + (T_qN)^{\perp_\omega}) = \dim M,
\end{equation} because $L$ is Lagrangian and $N$ coisotropic. Substituting (\ref{eq2symplectic})in (\ref{eq1symplectic}) we obtain \begin{align} \nonumber
\dim L_N &= \dim M - \dim (T_qL + (T_qN)^{\perp_\omega})- \dim (T_q L \cap (T_qN )^{\perp_\omega}) \\ & \nonumber = \dim M - \dim L - \dim (T_qN)^{\perp_\omega} = \dim M - \dim L -(\dim M - \dim N)\\ &\nonumber = \dim N - \dim L = \dim N - \frac{1}{2} \dim
M, \end{align} which is exactly $\frac{1}{2} \dim N/ \mathcal{F}$, as a direct calculation shows.
    \end{proof}
\end{frame}

\begin{frame}{Poisson bracket and Coisotropic submanifolds}

\begin{definition} For $f, g \in C^\infty(M)$, their \alert{Poisson bracket} is defined as $$\{f,g\} := \omega(X_f, X_g).$$
\end{definition}

We have the following characterization, which is fundamental for the theory of constraints.

\begin{proposition}A submanifold $i: N \rightarrow M$ is coisotropic if and only if, for every pair of functions, $f,g$ constant on $N$, $\{f,g\} = 0$ on $N$.
\end{proposition}
\end{frame}

\begin{frame}{Results to be generalized}
\begin{enumerate}
    \item Endowing $TM$ with the symplectic structure obtained from $\flat: TM \rightarrow T^\ast M$, we can interpret dynamics as Lagrangian submanifolds.
    \pause
    \item Coisotropic submanifolds can be reduced to a symplectic manifold.
    \pause
    \item Lagrangian submanifolds project onto Lagrangian submanifolds under this reduction
    \pause
    \item Coisotropic submanifolds can be characterized by the Poisson bracket.
\end{enumerate}    
\end{frame}