
Multisymplectic geometry is the natural framework in which to formulate classical field theories, just as symplectic geometry plays that key role in Lagrangian and Hamiltonian mechanics \cite{abraham2008foundations,ClassicalFieldsSniaiycki,dLR}. Indeed, the bundles of exterior forms are naturally equipped with a multisymplectic form, in the same way that for the bundle of $1$-forms (i.e. the cotangent bundle of the manifold) the natural structure is a symplectic form. However, multisymplectic geometry exhibits a much higher degree of complexity, dealing with differential forms of higher degree.  These differences make multisymplectic geometry richer but at the same time more complicated, and if the holy grail of classical field theories is to seek a full extension of the results in symplectic mechanics, this task is far from being fully achieved. This paper tries to cover some aspects that have already been partially dealt with in previous papers \cite{Ibort1999OnTG,HamiltonianStructuresIbort}, thus initiating an ambitious plan that we hope to complete in the coming years.\\

One of the key aspects of this new approach is not to consider any notion of regularity in the definition of a multisymplectic form, as is usually done in applications to classical field theories \cite{KdVequationGotay,GotayMultisymplecticFramework,MarkSpaceTimeDecomposition,Ibort1999OnTG,Roman_Roy_2009,NarcisoMultisymplecticFormalism,Invitation2019}. This allows us to work with greater flexibility, recovering regularity as a particular case. Our main objective in this paper is to study the submanifolds of a multisymplectic manifold, in particular the relations between Lagrangian and coisotropic submanifolds \cite{Ibort1999OnTG,deleon2003tulczyjews,Sevestre_2019}. In doing so, we prove a coisotropic reduction theorem which generalises the one already known for symplectic geometry. The interest of this reduction lies in the fact that the Lagrangian submanifolds are the geometric interpretation of the dynamics, and if one of them has a clean intersection with a coisotropic one, it can be reduced to the quotient of the latter while maintaining the Lagrangian character (and so, providing a reduced dynamics) \cite{WeinsteinLagrangianNeighborhood,abraham2008foundations}. Very relevant by-products of these notions and results are the construction of graded brackets and the interpretation of a coisotropic submanifold in terms of these brackets, as well as the study of currents and conserved quantities \cite{HamiltonianStructuresIbort,FORGER_2003,Blacker_2021} (see also \cite{AitorSymmetries,ConservedQuantitiesMarco}).
We would like to mention that the graded brackets that are used in this paper are related to the notion of higher-Poisson structures (see \cite{Bursztyn2015}), a generalization of the notion of a Poisson structure. \\

The paper is structured as follows. \cref{section:MultisymplecticGeometry} introduces the fundamental concepts of both the multilinear version of symplectic geometry in the realm of vector spaces and the corresponding translation to the realm of differentiable manifolds. In this section we introduce the main examples of multisymplectic vector spaces and multisymplectic manifolds. In the first case, we pay special atention to multisymplectic structures of forms arising from a vector space together with a ``vertical" subspace. Similarly, in the second case we study bundles of forms over a manifold together with a regular ``vertical" distribution. In \cref{Section:Hamiltonianstructures}, we develop the notion of Hamiltonian vector fields and Hamiltonian forms; it should be noticed that we do not ask for any regularity conditions from the multisimplectic forms, so we have to work with the respective kernels to avoid singularities. Thus we can interpret multivector fields as Lagrangian submanifolds of multisymplectic manifolds by naturally extending the results known in symplectic geometry. At the same time, we complete the results of previous work, which allow us to introduce a graded Lie algebra of brackets. We can also consider an abstract framework for the study of currents and conserved quantities. Finally, in \cref{section:CoisotropicSubmanifolds} we obtain the extension of the coisotropic reduction theorem as well as the reduction of Lagrangian submanifolds via coisotropic reduction. To do that, we need to extend some theorems on coisotropic manifolds due to Weinstein. The paper ends with some conclusions and a list of potential future work in \cref{conclusions}.