%&lualatex
%!TEX TS-program = lualatex
%!TEX encoding = UTF-8 Unicode
% !TeX spellcheck = es-ES
\documentclass{beamer}
\usepackage{graphicx}
\usepackage[english]{babel}
\usepackage{mathtools}
\usepackage{datetime}
\usepackage{amssymb}
\usepackage{physics}
%--- Cosas de estilo
\usetheme{metropolis}
\metroset{
    sectionpage=progressbar, 
    numbering=fraction, 
    progressbar=foot,
    block=fill
}
\usecolortheme{default}
\uselanguage{english}
\languagepath{english}
\setbeamercovered{transparent} %Para que las pausas bajen la opacidad de los elementos en vez de ocultarlos
%--- Para que las demostraciones no ocupen tanto
\setbeamertemplate{proof begin}{
    \vspace{0.5ex}%
    {\usebeamercolor[fg]{block title}\usebeamerfont*{block title}%
    \insertproofname\ }%
}
\setbeamertemplate{proof end}{\par
} %https://tex.stackexchange.com/a/537223/
\addto\captionsspanish{\renewcommand\proofname{Dem.}} %Más corto

%--- Más cosas de estilo (fuentes más legibles)
\usepackage[mathrm=sym,warnings-off={mathtools-colon,mathtools-overbracket}]{unicode-math}
\setsansfont[
    BoldFont={Fira Sans Medium}, 
    BoldItalicFont={Fira Sans Medium Italic}, 
    ItalicFont={Fira Sans Book Italic}
]{Fira Sans Book} %algo más fina que Fira Sans Regular

%\setmathfont[Scale = MatchUppercase]{FiraMath-Regular.otf}
%\setmathfont[Scale = MatchUppercase, range={cal, bfcal, scr, bfscr, frak, bffrak}]{texgyredejavu-math.otf} %caligráficas y etc

%--- Para tener un poco más de rango (si una palabra no cabe en la línea por poquito, la hace más estrecha sin que se note demasiado)
% Solo funciona con Lua(La)TeX, que en Overleaf tarda demasiado en compilar.

\usepackage{microtype}
\microtypesetup{stretch=25, nopatch=footnote}

%--- un apaño para que no cuente las diapositivas del apéndice https://tex.stackexchange.com/a/70495/
\newcommand{\backupbegin}{
   \newcounter{finalframe}
   \setcounter{finalframe}{\value{framenumber}}
}
\newcommand{\backupend}{
   \setcounter{framenumber}{\value{finalframe}}
}

%--- Un par de teoremas base

\theoremstyle{definition}
\newtheorem{proposition}[theorem]{Proposition}

\title{Coisotropic reduction in different phase spaces}
\subtitle{XVIII International Young Researchers Workshop in Geometry, Dynamics and Field Theory}
\author{Rubén Izquierdo, Manuel De León}
\institute{UCM-ICMAT}
\newdate{date}{21}{2}{2024}
\date{\displaydate{date}}

\begin{document}
{
    \vfuzz=16pt %así no se queja
    \frame[noframenumbering, plain]{\titlepage} %Sin número de página abajo a la derecha
}

\begin{frame}{Outline}
    \tableofcontents
\end{frame}
%COISOTROPIC REDUCTION IN NON-DISSIPATIVE MECHANICS
\section{Coisotropic reduction in non-dissipative mechanics}
\begin{frame}{The canonical phase spaces}
\begin{itemize}
    \item If $Q$ is the configuration space of a mechanical system, the phase space $M := T^\ast Q$ inherits a canonical \alert<1>{symplectic structure} $(M,\omega),$ $$\omega := \omega_Q = - d \lambda_Q = d q^i \wedge dp_i.$$ \pause
    \item The phase space $M := T^\ast Q \times \mathbb{R}$  inherits a canonical \alert<2>{cosymplectic structure}, $(M, \omega, \theta),$ $$\omega = \omega_Q = d q^i \wedge dp_i, \,\,\,\, \theta = dt.$$ 
\end{itemize}
\end{frame}
\begin{frame}{The Poisson bracket}
    Symplectic and cosympelctic manifolds are \alert<1>{Poisson manifolds} with the bracket $$\{ f , g\} = \pdv{f}{q^i} \pdv{g}{p_i} - \pdv{f}{p_i}\pdv{g}{q^i}.$$\pause In each of these cases, the bracket is induced by the \alert<2>{Poisson bivector} $$\Lambda = \pdv{q^i} \wedge \pdv{p_i}, \,\,\, \{f , g\} = \Lambda(df, dg).$$ \pause
    We have an induced map $$\sharp_\Lambda : T^\ast M \rightarrow TM, \,\,\,\,\, \alpha \mapsto \iota_\alpha \Lambda.$$\pause
    Denote $$\mathcal{H} := \operatorname{im} \sharp_\Lambda = \langle \pdv{q^i}, \pdv{p_i} \rangle.$$ In symplectic manifolds, $\mathcal{H} = TM.$
\end{frame}
\begin{frame}{Coisotropic and Lagrangian submanifolds}
If $\Delta \subseteq T_xM$, we define the \alert<1>{orthogonal} as $$\Delta^{\perp_\Lambda} := \sharp_\Lambda (\Delta ^0),$$ where $\Delta^0 \subseteq T^\ast_x M$ is the annihilator of $\Delta.$ \pause
We say that $\Delta$ is
\begin{itemize}
    \item \alert<2>{Coisotropic}, if $$\Delta^{\perp_\Lambda} \subseteq \Delta,$$ \pause
    \item \alert<3>{Lagrangian}, if $$\Delta^{\perp_\Lambda} = \Delta \cap \mathcal{H}.$$\pause
\end{itemize}
These definitions apply to submanifolds $N \hookrightarrow M$ as well.
\end{frame}
\begin{frame}{Coisotropic reduction in symplectic geometry}
Let $(M, \omega)$ be a \alert<1>{symplectic manifold} and $i:N \hookrightarrow M$ be a \alert<1>{coisotropic submanifold}. \pause
\begin{proposition}
$(TN)^{\perp_\Lambda} \subseteq TN$ is an involutive distribution.
\end{proposition}
\pause
Define $\mathcal{F}$ to be the maximal foliation associated to $(TN)^{\perp_\Lambda}$. We will assume that $N/ \mathcal{F}$ admits a manifold structure such that the canonical projection $\pi: N \rightarrow N/ \mathcal{F}$ is a summersion. \pause
\begin{theorem}[Weinstein] There exists an unique \alert<4>{symplectic} form $\omega_N$ defined on $N/ \mathcal{F}$ such that $$\pi^\ast \omega_N = i^\ast \omega.$$
\end{theorem}
\end{frame}
\begin{frame}{Coisotropic reduction in cosymplectic geometry}
Let $(M, \omega, \theta)$ be a \alert<1>{cosymplectic manifold} and $i: N \hookrightarrow M$ be a \alert<1>{coisotropic submanifold}. \pause
\begin{proposition}
$(TN)^{\perp_\Lambda} \subseteq TN$ is an involutive distribution.
\end{proposition}\pause
Suppose $N/ \mathcal{F}$ admits a manifold structure such that $\pi: N \rightarrow N/ \mathcal{F}$ defines a summersion.\pause
\begin{theorem}[ RIL-MLR]
\begin{itemize}
\item If $TN \subseteq \mathcal{H}$, $N/ \mathcal{F}$ admits an unique \alert<4>{symplectic structure} compatible with the strcuture defined on $M$.\pause
\item If $\pdv{t} \in TN$, $N/ \mathcal{F}$ admits an unique \alert<5>{cosymplectic sturcture} compatible with the one defined on $M$.
\end{itemize}
\end{theorem}
\end{frame}

%COISOTROPIC REDUCTION IN DISSIPATIVE MECHANICS
\section{Coisotropic reduction in dissipative mechanics}
\begin{frame}{The canonical phase spaces}
    \begin{itemize}
        \item The phase space of an autonomous dissipative system is $T^\ast Q \times \mathbb{R}$, with its canonical \alert<1>{contact structure} $$\eta = dz - p_i dq^i.$$ \pause
        \item If we want to study time-dependent dissipative mechanics, the phase space is $T^\ast Q \times \mathbb{R} \times \mathbb{R}$ endowed with its \alert<2>{cocontact structure}
        $$\eta = dz - p_i dq^i; \,\,\, \theta = dt.$$
    \end{itemize}
\end{frame}
\begin{frame}{The Jacobi bracket}
In both of these phase spaces there is a \alert<1>{Jacobi bracket} which is locally given by $$\{f,g\} =  \frac{\partial f}{\partial p_i}\frac{\partial g}{\partial q^i}- \frac{\partial f}{\partial q^i} \frac{\partial g}{\partial p_i} + p_i\left (  \frac{\partial f}{\partial p_i}\frac{\partial g}{\partial z}- \frac{\partial g}{\partial p_i} \frac{\partial f}{\partial z} \right ) + g \frac{\partial f}{\partial z} - f\frac{\partial g}{\partial z}.$$ \pause This Jacobi bracket is defined through the \alert<2>{Jacobi bivector} and a vector field 
$$\Lambda = \frac{\partial}{\partial p_i}\wedge \frac{\partial}{\partial q^i} + p_i \frac{\partial}{\partial p_i} \wedge \frac{\partial}{\partial z},$$
$$E = - \pdv{z},$$
as $$\{ f, g \} = \Lambda(df, dg) + f E(g) - gE(f).$$
\end{frame}
\begin{frame}{Coisotropic and Legendrian submanifolds}
The orthogonal of a distribution $\Delta \subseteq TM$ is defined as $$\Delta^{\perp_\Lambda} = \sharp_\Lambda(\Delta^0),$$ where $$\sharp_\Lambda : T^\ast M \rightarrow TM; \,\,\, \alpha \mapsto \iota_\alpha \Lambda.$$\pause
We say that $\Delta$ is
\begin{itemize}
    \item \alert<2>{Coisotropic}, if $$\Delta^{\perp_\Lambda} \subseteq \Delta,$$ \pause
    \item \alert<3>{Legendrian}, if $$\Delta^{\perp_\Lambda} = \Delta.$$\pause
\end{itemize}
The same definitions apply to submanifolds $N \hookrightarrow M$ as well.
\end{frame}
\begin{frame}{Coisotropic reduction}
\begin{proposition}
If $N \hookrightarrow M$ is a \alert<1>{coisotropic} submanifold, then $(TN)^{\perp_\Lambda}$ is involutive and thus arises from a maximal foliation $\mathcal{F}$.
\end{proposition}
We assume that $\pdv{z} \in TN$. \pause
\begin{theorem}
If $M$ is a \alert<2>{contact} manifold, $N/ \mathcal{F}$ admitis an unique \alert<2>{contact} structure compatible with the one on $M$. \\
\pause
If $M$ is a \alert<3>{cocontact} manifold:\pause
\begin{itemize}
\item If $\pdv{t} \in TN,$ $N/\mathcal{F}$ inherits an unique \alert<4>{cocontact} structure from $M$. \pause
\item If $TN \subseteq \operatorname{im} \sharp_\Lambda \oplus \langle \pdv{z} \rangle$, $N/ \mathcal{F}$ inherits an unique \alert<5>{contact} structure from $M$.
\end{itemize}
\end{theorem}
\end{frame}
\begin{frame}{References}
    \bibliographystyle{plain}
    \bibliography{refs.bib}
\end{frame}


\end{document}
