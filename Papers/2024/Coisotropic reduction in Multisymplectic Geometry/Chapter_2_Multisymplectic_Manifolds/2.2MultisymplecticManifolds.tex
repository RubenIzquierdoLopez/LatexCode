\begin{Def}[Multisymplectic manifold] A \textbf{multisymplectic manifold} of order $k$ is a pair $(M, \omega)$, where $M$ is a manifold, and $\omega$ is a \textbf{multisymplectic form} of order $k$, that is, a closed $(k +1)$-form. $(M, \omega)$ will be called \textbf{non-degenerate} if $\omega_x$ is non-degenerate for every $x \in M$.
\end{Def}

The $j$th-orthogonal complement defined in \cref{Subsection:MultisymplecticVectorSpapces} and the notion of $j$-isotropic, $j$-coisotropic, $j$-Lagrangian and regular subspaces generalizes to distributions $\Delta$, and submanifolds $N$, defining it in each subspace $\Delta_x$, or in each tangent space $T_x N.$

\begin{example} \label{ex:formsmanifold}We can generalize \cref{ex:canonicalmultisymplectic} to manifolds. Fix a manifold $L$ and define $$M:= \bigwedge^k L,$$ the bundle of $k$-forms. We have the tautological $k$-form $$\Theta^k_L|_{\alpha_x}(v_1, \dots, v_k):= \alpha_x(\pi_\ast v_1, \dots, \pi_\ast v_k),$$ where $\pi: \bigwedge^k L \rightarrow L$ is the natural projection. Define $$\Omega^k_L := - d\Theta^k_L.$$ Then $(\bigwedge^k L, \Omega^k_L)$ is a non-degenerate multisymplectic manifold of order $k$. It is easy to check (see \cref{lemma:completelift}) that $\Theta^k_L$ and $\Omega^k_L$ are the only forms on $\bigwedge^k L$ satisfying $$\alpha^\ast \Theta^k_L = \alpha, \,\, \alpha^\ast \Omega^k_L = - d \alpha,$$ for every $k$-form (interpreted as a section) $\alpha: L \rightarrow \bigwedge^k L$.
\end{example}

 In canonical coordinates $(x^i, p_{i_1, \dots, i_k})$ on $\bigwedge^k L$, we have $$\Theta^k_L = p_{i_1, \dots, i_k} d x^{i_1} \wedge \cdots \wedge dx^{i_k},$$ and $$\Omega^k_L = -dp_{i_1, \dots, i_k} \wedge d x^{i_1} \wedge \cdots \wedge dx^{i_k}.$$ This immediately shows that the vertical distribution $W_L^k$ associated to the vector bundle $\bigwedge^k L \rightarrow L$ is $1$-isotropic. Additionally, we have:

 \begin{proposition}\label{prop:KLagrangianclosed} $W_L^k$ defines a $1$-Lagrangian distribution. Furthermore, a form (interpreted as a section) $$\alpha: L \rightarrow \bigwedge^k L$$ defines a $k$-Lagrangian submanifold if and only if it is closed.
 \end{proposition}
 \begin{proof} By \cref{isotropicdecomposition}, it is enough to show that $\alpha$ defining a $k$-isotropic submanifold is equivalent to $\alpha$ being closed (this would imply that $\alpha(L)$ is $k$-Lagrangian, and that $W^k_L$ is $1$-Lagrangian, since they are complementary). Indeed, by \cref{prop:characterizationisotropic}, $\alpha(L)$ is $k$-isotropic if and only if $$ 0 = \alpha^\ast \Omega^k_L = - d \alpha, $$ that is, if and only if $\alpha$ is closed.
 \end{proof}

 There is another relevant type of multisymplectic manifolds that generalizes \cref{ex:verticalforms}:
 \begin{example}\label{ex:verticalformmanifold} Let $L$ be a manifold and $\mathcal{E}$ be a regular distribution, where $r, k, \mathcal{E}_x, T_x L$ are in the hypotheses of \cref{hypotheseskr} for every $x \in L$. Define $$\bigwedge^k_r L:= \left\{ \alpha_x \in \bigwedge^k T^\ast_xL: \,\, \iota_{e_1 \wedge \cdots \wedge e_r} \alpha_x = 0,\,\, \forall e_1, \dots, e_r \in \mathcal{E}_x\right\}.$$ It is easy to check that $\bigwedge^k_r L$ defines a non-singular submanifold of $\bigwedge^k L$. Therefore, $$\left(\bigwedge^k_r L, \Omega_L\right)$$ is a multisymplectic manifold of order $k$.
 \end{example}

 \begin{remark}Just like in \cref{Subsection:MultisymplecticVectorSpapces}, throughout the rest of the text we will assume the conditions that make $\bigwedge^k_r L$ a regular multisymplectic manifold.
 \end{remark}

 A natural question to ask is what are the necessary (and sufficient) conditions for a multisymplectic manifold $(M, \omega)$ to be locally multisymplectomorphic to either of the models presented in \cref{ex:formsmanifold} or in \cref{ex:verticalformmanifold}. Of course, if it were the case, the multisymplectic vector space $(T_x M, \omega_x)$ would necessarily be of type $(k, r)$ (for the corresponging values in the model).
 
 \begin{Def}[\cite{deleon2003tulczyjews} Multisymplectic manifold of type $(k,r)$] A multisymplectic manifold of type $(k,r)$ is a tuple $(M, \omega, W, \mathcal{E})$, where $(T_xM,\omega_x, W_x, \mathcal{E}_x)$ is a multisymplectic vector space of type $(k,r)$ and $W$ is a regular integrable distribution.
 \end{Def}
 In \cite{Martin1988ADT}, G. Martin gave the characterization for multisymplectic manifolds of type $(k, 0)$.
   \begin{theorem}[\cite{Martin1988ADT} Darboux theorem for multisymplectic manifolds of type $(k,0)$] Let $(M, \omega, W)$ be a multisymplectic manifold of type $(k,0)$. Then, around each point $x \in M$ there exists a neighborhood $U$ of $x$ in $M$, a manifold $L$, and a multisymplectomorphism $$\phi: (U, \omega) \rightarrow (V, \Omega_L)$$ where $V$ is an open subset of $\bigwedge^k L $. 
 \end{theorem}
 And, in \cite{deleon2003tulczyjews}, M. de León et. al. generalized the result to multisymplectic manifolds of type $(k, r)$.
   \begin{theorem}[\cite{deleon2003tulczyjews} Darboux theorem for multisymplectic manifolds of type $(k,r)$] \label{thm:darbouxmultisymplectic} Let $(M, \omega, W, \mathcal{E})$ be a multisymplectic manifold of type $(k,r)$. Then, around each point $x \in M$ there exists a neighborhood $U$ of $x$ in $M$, a manifold $L$, and a multisymplectomorphism $$\phi: (U, \omega) \rightarrow (V, \Omega_L)$$ where $V$ is an open subset of $\bigwedge^k_r L $.
   \end{theorem}

   For a recent review on ``Darboux type Theorems'' in geometric structures appearing in the geometric formulation of Classical Field Theories, we refer to \cite{gràcia2023darboux}.