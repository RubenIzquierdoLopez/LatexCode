In this paper we have analysed the role that Lagrangian and coisotropic submanifolds play in multisymplectic geometry, with the intention of extending as far as possible the well-known results in symplectic geometry. When dealing with forms of degree higher than 2, there are different complements to a submanifold, which enriches the geometry but at the same time makes it more complex. One of the first results obtained is the interpretation of Lagrangian submanifolds as possible dynamics, as well as the introduction of a graded bracket algebra. This makes it possible to deal with currents and conserved quantities. The main result of the paper is a coisotropic reduction theorem which we hope will be useful in applications to multisymplectic field theory.\\

In future work we have proposed the following objectives:

\begin{enumerate}

\item Apply the results obtained in the current paper to multisymplectic field theories.

\item Since some field theories are singular, we would like to develop a regularization method as in the case of singular Lagrangian dynamics (see \cite{ibort-marin}); previously, we have to prove a coisotropic embedding theorem \'a la Gotay \cite{gotay,zambon} in the context of multisymplectic geometry.

\item  Develop the covariant approach through a space-time decomposition, and interpret the coisotropic reduction in the corresponding infinite dimensional setting.

\item Following the notion of multi-Dirac (and higher Dirac) structures \cite{joris_MultiDirac1,jorisDirac2, ZambonDirac, BursztynDirac}, we would like to develop a more general extension using the graded Poisson brackets defined in the current paper.

\item Extend the results to the realm of multicontact geometry (see \cite{multicontacto}).

\end{enumerate}

\section*{Acknowledgements}

We acknowledge the financial support of Grant PID2022-137909NB-C2, the Severo Ochoa Programme for Centres of Excellence in R\&D (CEX2019-000904-S), and Severo Ochoa scholarship for master students. 