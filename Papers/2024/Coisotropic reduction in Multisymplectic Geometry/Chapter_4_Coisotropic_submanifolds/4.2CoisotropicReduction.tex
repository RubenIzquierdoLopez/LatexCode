%-------------------------------------------
%Coisotropic reduction
%-------------------------------------------
When $k = 1$, that is, when $(M, \omega)$ is a symplectic manifold, we have the classical result of coisotropic reduction due to Weinstein \cite{weinstein1977lectures}.

\begin{theorem}\label{CoisotropicReductionSymplectic}[Coisotropic reduction in symplectic geometry] Let $(M, \omega)$ be a symplectic manifold, $i: N \hookrightarrow M$ be a coisotropic submanifold, and $j: L \hookrightarrow M$ be a Lagrangian submanifold that has clean intersection with $N$.  Then, $TN^{\perp}$ is an integrable distribution and determines a foliation $\mathcal{F}$ of maximal integral leaves. Suppose that the quotient space $N/ \mathcal{F}$ admits an smooth manifold structure such that the canonical projection $$\pi: N \rightarrow N/\mathcal{F}$$ defines a submersion. Then there exists an unique symplectic form on $N/\mathcal{F}$, $\omega_N$ compatible with $\omega$ in the following sense $$\pi^\ast \omega_N = i^\ast \omega.$$ Furthermore, if $\pi(N \cap L)$ is a submanifold, it is Lagrangian in $(N/\mathcal{F}, \omega_N).$
\end{theorem}

We would like to find an analogous result in multisymplectic manifolds. For the first part of \cref{CoisotropicReductionSymplectic}, the classical argument works.

\begin{proposition}[\cite{Ibort1999OnTG}]\label{Prop:Involutivity} Let $(M, \omega)$ be a multisymplectic manifold of order $k$ and $i:N\hookrightarrow M$ be a \kcoiso submanifold. Then, $(TN)^{\perp, k} \cap TN \subseteq TN$ defines an involutive distribution.
\end{proposition}
\begin{proof} Let $X, Y \in \mathfrak{X}(N)$ be vector fields on $N$ with values in $(TN)^{\perp, k},$ and let $Z_1, \dots, Z_k \in \mathfrak{X}(N)$ be arbitrary vector fields on $N$. Denote $$\omega_0 := i^\ast \omega.$$ Since $\omega$ is closed, we have
\begin{align*}
    0 &= (d \omega_0)(X, Y, Z_1, \dots, Z_k) =  X(\omega_0(Y, Z_1, \dots, Z_k)) - Y(\omega_0(X, Z_1, \dots, Z_k))\\
    &+ \sum_{j = 1}^k (-1)^j Z_i(\omega_0(X, Y, Z_1 \dots, \hat{Z}_i, \dots, Z_k)) - \omega_0([X, Y], Z_1, \dots, Z_k)\\
    &+ \sum_{j = 1}^k (-1)^{j +1}\omega_0([X, Z_i], Y, Z_1 \dots, \hat{Z}_i, \dots, Z_k)\\
    &+ \sum_{j = 1}^k (-1)^{j}\omega_0([Y, Z_i], X, Z_1 \dots, \hat{Z}_i, \dots, Z_k)\\
    &+ \sum_{i < j} (-1)^{i +j} \omega_0([Z_i, Z_j], X, Y, Z_1 \dots, \hat{Z}_i, \dots,\hat{Z}_j, \dots, Z_k).
\end{align*}
Now, since both $X$ and $Y$ take values in $(TN)^{\perp, k}$, all the summands but $$\omega_0([X, Y], Z_1, \dots, Z_k)$$ are zero. Therefore, we conclude $$\omega_0([X, Y], Z_1, \dots, Z_k) = 0,$$ for all $Z_1, \dots, Z_k \in \mathfrak{X}(N),$ that is, $[X, Y]$ takes values in $(TN)^{\perp, k}$, proving that the distribution is involutive.
\end{proof}

If $(TN)^{\perp, k} \cap TN$ is regular, by Frobenius' Theorem, it determines a foliation $\mathcal{F}$ of maximal leaves. We have the following result.

\begin{theorem}[\cite{Ibort1999OnTG}]\label{CoisotropicReductionMultisymplectic} Let $(M, \omega)$ be a multisymplectic manifold of order $k$, and $i: N \hookrightarrow M$ be a \kcoiso submanifold such that $(TN)^{\perp, k} \cap TN$ is regular. Suppose that $N/\mathcal{F}$ admits a smooth manifold structure such that the canonical projection $$\pi: N \rightarrow N/\mathcal{F}$$ defines a submersion. Then there exists an unique multisymplectic form of order $k$ on $N/\mathcal{F}$, $\omega_{N}$, that is compatible with $\omega$, that is, $$\pi^\ast \omega_N = i ^\ast \omega.$$
\end{theorem}
\begin{proof} Let $x \in N$. Notice that, since $\pi$ defines a submersion, we have the identification $$T_{[x]} N/\mathcal{F} = T_x N /\ker d_x \pi = T_xN/ (T_xN)^{\perp, k} \cap T_xN.$$ Let $v_1, \dots, v_{k+1} \in T_x N.$ The relation $\pi^\ast \omega_N = i^\ast \omega$ forces us to define $$\omega_N |_{[x]}([v_1], \dots, [v_{k+1}]) := \omega|_x(v_1, \dots, v_{k+1}),$$ proving that $\omega_N$ is unique. It only remains to show that the previous definition does not depend on the choice of $x$ and $v_i$. For the latter, first observe that if $[v] = 0$, that is, $v \in (T_xN)^{\perp, k}$ we have $$\omega(v, v_1, \dots, v_k) = 0,$$ for all $v_1, \dots, v_k \in T_xN.$ Therefore, if $[v_i] = [u_i],$ for $i = 1, \dots, k+1$, we have
\begin{align*}
    \omega_N|_{[x]}([v_1], \dots, [v_{k+1}]) &= \omega|_x (v_1, \dots, v_{k+1}) = \omega|_x(u_1, v_2, \dots, v_{k+1}) = \dots \\
    &= \omega|_x(u_1, \dots, u_{k+1}) = \omega_N|_{[x]}([u_1], \dots, [u_{k+1}]).
\end{align*}
For the independence of the chosen point, given $x, y \in N$ in the same leaf, we can find a complete vector field $X$ on $N$ with values in $(TN)^{\perp, k}$ such that its flow satisfies $$\phi^X_1(x) = y.$$ Now, denoting $\omega_0 := i^\ast \omega$, we have $$\pounds_X \omega_0 = \iota_X d \omega_0 + d \iota_X \omega_0 = 0,$$ since $\omega_0$ is closed and $\iota_X \omega_0 = 0$ (given that $X$ takes values in $(TN)^{\perp, k}$). This implies $(\phi^X_1) ^\ast \omega_0 = \omega_0.$ In particular, given $v_1, \dots, v_{k+1} \in T_x N$ we have
\begin{align*}
    \omega_N |_{[x]}([v_1], \dots, [v_{k+1}]) &= \omega_0 |_x(v_1, \dots, v_{k+1}) = \omega_0 |_y(d_x \phi^X_1 \cdot v_1, \dots, d_x\phi^X_1 \cdot v_{k+1})\\
    &= \omega_N|_{[y]}([d_x \phi^X_1 \cdot v_1], \dots, [d_x\phi^X_1 \cdot v_{k+1}]).
\end{align*}
Since $X$ is tangent to $\mathcal{F}$, its flow $\phi^X_1$ leaves invariant the foliation, and $\pi \circ \phi = \pi$. In particular, $$[v_i] = d_x \pi \cdot v_i = d_y \pi \cdot d_x \phi \cdot v_i = [d_x \phi \cdot v_i].$$ Finally, if $v_1, \dots, v_{k+1} \in T_x N$, $u_1, \dots, u_{k+1} \in T_yN$ with $[v_i] = [u_i]$,
\begin{align*}
    \omega_N|_{[x]}([v_1], \dots, [v_{k+1}]) &= \omega_0|_x(v_1, \dots, v_{k+1}) = \omega_0|_y(d_x \phi^X_1 \cdot v_1, \dots, d_x\phi^X_1 \cdot v_{k+1})\\
    &= \omega_N|_{[y]}([d_x \phi^X_1 \cdot v_1], \dots, [d_x\phi^X_1 \cdot v_{k+1}])\\
    &= \omega_N|_{[y]}([u_1], \dots, [u_{k+1}]),
\end{align*}
proving the result.
\end{proof}

For the projection of Lagrangian submanifolds, the second part of \cref{CoisotropicReductionSymplectic}, multisymplectic manifolds are \textit{too general} and hard to study without asking for further structures. Indeed, we can easily find a counterexample.
\begin{example}[A counterexample] Let $L = \langle l_1, l_2, l_3\rangle$ be a $3$-dimensional vector space and define $$V:= L \oplus \bigwedge^2 V^\ast.$$ Let $l^1, l^2, l^3$ be the dual basis induced on $L^\ast$ and denote $$\alpha^{ij} := l^i \wedge l^j.$$ Then $$V = \langle l_1, l_2, l_3, \alpha^{12}, \alpha^{13}, \alpha^{23}\rangle.$$ Let $l^1, l^2, l^3, \alpha_{12}, \alpha_{13}, \alpha_{23}$ be the dual basis. We have $$\Omega_L = \alpha_{12} \wedge l^1 \wedge l^2 + \alpha_{13} \wedge l^1 \wedge l^3 + \alpha_{23}\wedge l^2 \wedge l^3.$$ Define $$N := \langle l_1 + l_2, l_1 + \alpha^{23}, l_2 + \alpha^{13}, l_3, \alpha^{12}\rangle.$$ Then $N$ is a $2$-coisotropic subspace. Indeed, a quick calcultion shows $N^{\perp, 2} = 0$. This implies that the quotient space $N/N^{\perp, 2}$ is (isomorphic to) $N.$ Now, taking as the $2$-Lagrangian subspace $L = \langle l_1, l_2,l_3\rangle$, we have $$L \cap N = \langle l_1 + l_2 , l_3 \rangle.$$ However, this does not define a $2$-Lagrangian subspace of $(N, \Omega_L |_N)$, since $\alpha^{12} \in (N \cap L)^{\perp, 2}$, but $ \alpha^{12} \not \in(L \cap W).$
\end{example}
Nervertheless, we will be able to find a generalization of the previous theorem restricting the study to a particular class, those that locally are bundles of forms, which are precisely the multisymplectic manifolds appearing in classical field theories \cite{gotay2004momentum}. More particularly, we will study coisotropic reduction of vertical coisotropic submanifolds in multisymplectic manifolds of type $(k,r)$.\\

The classical proof of the last part of \cref{CoisotropicReductionSymplectic} uses en elaborate comparison of dimensions argument (see \cite{abraham2008foundations}). This argument hardly translates to multisymplectic manifolds since, in general, the map $$TM \xrightarrow{ \flat_1} \bigwedge^k M$$ does not define a bundle isomorphism. However, we can prove it using the local form proved in \cref{LocalFormSection}.\\

\label{NomalFormCoisotropicReduction} Given some manifold $L$, and a regular distribution on $L$, $\mathcal{E}$, define $$M := \bigwedge^k_r L$$ endowed with its canonical multisymplectic structure. Here, the horizontal forms, are taken with respect to $\mathcal{E}$. Let $i: Q \hookrightarrow L$ be a submanifold of dimension at least $k$ (for $\bigwedge^k Q$ to be non-zero) and take $$N:= \bigwedge^k_rL \big |_Q$$ the restricted bundle to $Q$. Then, $N \hookrightarrow M$ is a  \kcoiso submanifold. Indeed, under the (non-canonical) identification $$T_{(x, \alpha)}N = T_xQ \oplus \bigwedge^k_r T^\ast_xL,$$ for $(x, \alpha) \in N$, we have $$(TN)^{\perp, k} = 0 \oplus \ker i^\ast,$$ where $i^\ast$ is the induced map $$i^\ast: \bigwedge^k_r T^\ast_x L \subseteq \bigwedge^k T^\ast_x L \rightarrow \bigwedge^k T^\ast_x Q.$$ We claim that the image of $\bigwedge^k_r T^\ast_x L$ under $i^\ast$ is $\bigwedge ^k_r T^\ast_x Q$, where the horizontal forms are taken with respect to the subspace $$\widetilde{\mathcal{E}}_x := \mathcal{E}_x \cap T_xQ.$$ Indeed, it is clear that $$i^\ast \left(\bigwedge^k_r T^\ast_xL \right) \subseteq \bigwedge^k_r T^\ast_x Q,$$ since, if $e_1, \dots, e_r \in \widetilde{\mathcal{E}}_x$ and $\alpha \in \bigwedge^k_r T^\ast_xL$, we have $$\iota_{e_1 \wedge \dots \wedge e_r} i ^\ast \alpha = i^\ast(\iota_{e_1 \wedge \dots \wedge e_r} \alpha) = 0.$$ Now, to see the other inclusion, we take a projection $$p: T_xL \rightarrow T_xQ$$ that satisfies $p(\mathcal{E}_x) = \widetilde{\mathcal{E}}_x,$ that is, a projection that makes the following diagram commutative
% https://q.uiver.app/#q=WzAsNCxbMCwwLCJcXG1hdGhjYWx7RX1feCJdLFswLDEsIlxcd2lkZXRpbGRle1xcbWF0aGNhbHtFfX1feCJdLFsyLDEsIlRfeFEiXSxbMiwwLCJUX3hMIl0sWzAsMywiIiwwLHsic3R5bGUiOnsidGFpbCI6eyJuYW1lIjoiaG9vayIsInNpZGUiOiJ0b3AifX19XSxbMSwyLCIiLDAseyJzdHlsZSI6eyJ0YWlsIjp7Im5hbWUiOiJob29rIiwic2lkZSI6InRvcCJ9fX1dLFswLDEsInB8X3tcXG1hdGhjYWx7RX1feH0iXSxbMywyLCJwIl1d
\[\begin{tikzcd}
	{\mathcal{E}_x} && {T_xL} \\
	{\widetilde{\mathcal{E}}_x} && {T_xQ}
	\arrow[hook, from=1-1, to=1-3]
	\arrow[hook, from=2-1, to=2-3]
	\arrow["{p|_{\mathcal{E}_x}}", from=1-1, to=2-1]
	\arrow["p", from=1-3, to=2-3]
\end{tikzcd}.\]
Take $\beta \in \bigwedge^k_r T_x ^\ast Q$ and define $\alpha \in \bigwedge^k T^\ast_xL$ as $$\alpha:= p^\ast \beta.$$ It is clear that $i^\ast \alpha = \beta.$ Furthermore, since $p$ satisfies $p(\mathcal{E}_x) = \widetilde{\mathcal{E}}_x,$ we have $$\alpha \in \bigwedge^k_r T^\ast_xL,$$ proving that $$i^\ast \left(\bigwedge^k_r T^\ast_xL \right) = \bigwedge^k_r T^\ast_x Q.$$ In particular, when $\mathcal{E} \cap TQ$ has constant rank, so does\footnote{Because $\ran (TN)^{\perp, k} = \ran \ker i^\ast = \ran \bigwedge^k_r L - \ran \bigwedge^k_r Q$} $(TN)^{\perp, k}$, and we have that the maximal integral leaf of this distribution that contains $(x,0)$ is $$\ker i^\ast |_{\{x\}} = \{(x, \alpha): \alpha \in \ker i^\ast, \alpha \in \bigwedge^k_r T^\ast_x L\}.$$ These leaves define a vector subbundle
% https://q.uiver.app/#q=WzAsMyxbMCwwLCJcXGtlciBpXlxcYXN0Il0sWzIsMCwiXFxiaWd3ZWRnZV5rX3JMXFxiaWd8X1EiXSxbMSwxLCJRIl0sWzAsMSwiIiwyLHsic3R5bGUiOnsidGFpbCI6eyJuYW1lIjoiaG9vayIsInNpZGUiOiJ0b3AifX19XSxbMCwyXSxbMSwyXV0=
\[\begin{tikzcd}
	{\ker i^\ast} && {\bigwedge^k_rL\big|_Q} \\
	& Q
	\arrow[hook, from=1-1, to=1-3]
	\arrow[from=1-1, to=2-2]
	\arrow[from=1-3, to=2-2]
\end{tikzcd}.\]
By the previous considerations, these bundles fit in a short exact sequence
% https://q.uiver.app/#q=WzAsNSxbMSwwLCJcXGtlciBpXlxcYXN0Il0sWzIsMCwiXFxiaWd3ZWRnZV5rX3JMIl0sWzMsMCwiXFxiaWd3ZWRnZV5rX3JRIl0sWzQsMCwiMCJdLFswLDAsIjAiXSxbMCwxLCIiLDAseyJzdHlsZSI6eyJ0YWlsIjp7Im5hbWUiOiJob29rIiwic2lkZSI6InRvcCJ9fX1dLFsxLDIsImleXFxhc3QiXSxbMiwzXSxbNCwwXV0=
\[\begin{tikzcd}
	0 & {\ker i^\ast} & {\bigwedge^k_rL} & {\bigwedge^k_rQ} & 0
	\arrow[hook, from=1-2, to=1-3]
	\arrow["{i^\ast}", from=1-3, to=1-4]
	\arrow[from=1-4, to=1-5]
	\arrow[from=1-1, to=1-2]
\end{tikzcd}.\]
Therefore, we may identify $$N/\mathcal{F} = \bigwedge^k_r Q,$$ where the horizontal forms are taken with respect to $\widetilde{\mathcal{E}} = \mathcal{E}\cap TQ$ (which we are assuming to have constant rank). A routine check shows that the \ms structure induced from \cref{CoisotropicReductionMultisymplectic} is none other than the canonical \ms structure on $\bigwedge^k_rQ.$\\

Now, let us study the projection of Lagrangian submanifolds. An important class of $k$-Lagrangian submanifolds in $\bigwedge^k_r L$ are given by closed forms (\cref{prop:KLagrangianclosed}) $$\alpha: L \rightarrow \bigwedge^k_r L.$$ We have the following diagram 
% https://q.uiver.app/#q=WzAsNSxbMiwwLCJcXGJpZ3dlZGdlXmtfciBMID0gTSJdLFswLDAsIkwiXSxbMCwxLCJRIl0sWzIsMSwiXFxiaWd3ZWRnZV5rX3JMIFxcYmlnfF9RID1OIl0sWzIsMiwiXFxiaWd3ZWRnZV5rX3IgUSA9IE4vXFxtYXRoY2Fse0Z9Il0sWzEsMCwiXFxhbHBoYSJdLFsyLDEsIiIsMCx7InN0eWxlIjp7InRhaWwiOnsibmFtZSI6Imhvb2siLCJzaWRlIjoidG9wIn19fV0sWzIsMywiXFxhbHBoYSB8X1EiXSxbMywwLCIiLDIseyJzdHlsZSI6eyJ0YWlsIjp7Im5hbWUiOiJob29rIiwic2lkZSI6InRvcCJ9fX1dLFszLDQsImkgXlxcYXN0ID0gXFxwaSJdLFsyLDQsImleXFxhc3QgXFxhbHBoYSIsMix7ImN1cnZlIjoyLCJzdHlsZSI6eyJib2R5Ijp7Im5hbWUiOiJkYXNoZWQifX19XV0=
\[\begin{tikzcd}
	L && {\bigwedge^k_r L = M} \\
	Q && {\bigwedge^k_rL \big|_Q =N} \\
	&& {\bigwedge^k_r Q = N/\mathcal{F}}
	\arrow["\alpha", from=1-1, to=1-3]
	\arrow[hook, from=2-1, to=1-1]
	\arrow["{\alpha |_Q}", from=2-1, to=2-3]
	\arrow[hook, from=2-3, to=1-3]
	\arrow["{i ^\ast = \pi}", from=2-3, to=3-3]
	\arrow["{i^\ast \alpha}"', curve={height=12pt}, dashed, from=2-1, to=3-3]
\end{tikzcd}.\]
It is clear that the projection of $\alpha(L) \cap N$ onto $N/\mathcal{F} = \bigwedge^k_r Q$ is exactly the image of $$i^\ast \alpha: Q \rightarrow \bigwedge^k_r Q.$$ Since $\alpha$ is closed, so is $i^\ast \alpha$, proving that in this local form, $k$-Lagrangian submanifolds complementary to the vertical distribution $\mathcal{W}$ reduce to $k$-Lagrangian submanifolds. Therefore, using \cref{localformCoisotropicLagrangian} we have the main result of this section:
\begin{theorem} Let $(M, \omega, \mathcal{W}, \mathcal{E})$ be a multisymplectic manifold of type $(k,r)$, $i:N \hookrightarrow M$ a $k$-coisotropic submanifold satisfying $$\mathcal{W} \big |_N \subseteq TN,$$ and $j:L \hookrightarrow M$ a \klagran submanifold complementary to $\mathcal{W}.$ Suppose that $N/\mathcal{F}$ admits a smooth manifold strcuture such that $\pi : N \rightarrow N/\mathcal{F}$ defines a submersion, where $\mathcal{F}$ is the foliation associated to $(TN)^{\perp, k}$ (see \cref{CoisotropicReductionMultisymplectic}), and that $$\mathcal{E} \big |_N \cap \left( TN/ \mathcal{W}\big |_N\right)$$ has constant rank. Then, if $\pi(L \cap N)$ is a submanifold, it is \klagran.
\end{theorem}


