%-------------------------------------------
%Local form of coisotropic submanifolds
%-------------------------------------------
Weinstein gave the first normal form\footnote{Along this paper, we reserve the term \textit{normal form} for a classification of a neighborhood of an entire submanifold (like in \cref{NeighborhoodTheorem}, \cref{LagrangianFormMultisymplectic}), and we use the term \textit{local form} for a classification of a neighborhood around any point of a submanifold (like in \cref{localformcoisotropic}).} theorem for Lagrangian submanifolds in the context of symplectic geometry.

\begin{theorem}[\cite{WeinsteinLagrangianNeighborhood} Weinstein's Lagrangian neighborhood Theorem]\label{NeighborhoodTheorem} Let $(M, \omega)$ be a symplectic manifold and $L \hookrightarrow M$ be a Lagrangian submanifold. Then there are neighborhoods $U$, $V$ of $L$ in $M$, and in $T^\ast L$ (identifying $L$ with the zero section) respectively, and a symplectomorphism $$\phi: U \rightarrow V.$$
\end{theorem}

This result has been generalized to multisymplectic manifolds of type $(k,0)$ by G. Martin \cite{Martin1988ADT}, and extended to multiysmplectic manifolds of type $(k,r)$ by M. de Leon et al. \cite{deleon2003tulczyjews}.

\begin{theorem}[\cite{deleon2003tulczyjews}]\label{LagrangianFormMultisymplectic}\label{normalformlagrangian} Let $(M, \omega, W, \mathcal{E})$ be a multisymplectic manifold of type $(k,r)$, and $L \hookrightarrow M$ be a $k$-Lagrangian submanifold complementary to $W$, that is, such that $$TL \oplus W \big |_L = TM \big |_L.$$ Then there are neighborhoods $U$, $V$ of $L$ in $M$, and of $L$ in $\bigwedge^k_r L$ (identifying $L$ as the zero section), where the horizontal forms are taken with respect to $\mathcal{E}$ under the identification $$TL = TM/ W,$$ and a multisymplectomorphism $$\psi: U \rightarrow V,$$ which is the identity on $L$ and satisfies $$\psi_\ast W = W^k_L,$$ where $W_L^k$ denotes the vertical distribution on $\bigwedge^k_r L^\ast.$
\end{theorem}

\begin{proof} Define the vector bundle isomorphism $$\phi: W|_L \rightarrow \bigwedge^k_r L; \,\, \phi(w_l) := (\iota_{w_l} \omega) |_L.$$ By the tubular neighborhood theorem, we may identify a neighborhood $U$ of $L$ in $W|_L$ with a neighborhood of $L$ in $M$. Under the previous identificaction, let $V := \phi(U)$ and define $$\widetilde{\omega} := \phi_\ast \omega.$$ Following the same line of reasoning as in \cref{prop: characterization}, we have $\widetilde{\omega} = \Omega^k_L$ on $L$. Furthermore, since $\phi$ is a vector bundle isomorphism, $\phi$ preserves fibers and we have $\phi_\ast W|_U = (W_L^k)|_V.$ This implies that $W_L^k$ not only defines a $1$-isotropic distribution for $\Omega^k_L$, but also for $\widetilde{\omega}.$ To build the multisymplectomorphism $\psi$, we will make use of \textit{Moser's trick} with the family of forms $$\Omega_t:= (1-t) \Omega_L^k + t \widetilde \omega.$$ More precisely, we will look for a time dependent vector field $X_t$ on $V$ such that its flow $\phi_t$ satisfies $$\phi_t ^\ast \Omega_t = \Omega^k_L,$$ for every $t$. To achieve this, it will be sufficient to look for a time dependent vector field $X_t$ such that $$0 = \dv{t}\left(\phi^\ast_t \Omega_t\right) = \pounds_{X_t} \Omega_t + \dv{\Omega_t}{t} = d \iota_{X_t} \Omega_t + \widetilde \omega - \Omega^k_L.$$ Now, if we denote by $\pi_t$ multiplication by $t$ in $\bigwedge^k_r L$, by reducing neighborhoods if necessary, we get a well defined map $$\pi_t : V \rightarrow V,$$ for $0 \leq t \leq 1$. By the relative Poincaré Lemma, we have $$\widetilde \omega = d \left( \int_0^1 \pi_t^\ast \iota_\Delta \widetilde{\omega} dt\right),$$ where $\Delta$ is the dilation vector field. Therefore, if we define $$\widetilde \theta := - \int_0^1 \pi_t^\ast \iota_{\Delta} \widetilde{\omega} dt,$$ it follows that $\widetilde \omega = - d \widetilde \theta,$ where $\widetilde \theta = 0$ on $L$ (because $\Delta = 0$ on $L$). Since we need $ \Omega ^k_L - \widetilde \omega = - d (\Theta^k_L - \widetilde \theta) = d\iota_{X_t} \Omega_t,$ it will be enough to look for $X_t$ satisfying $$\iota_{X_t} \Omega_t =  \widetilde \theta - \Theta^k_L.$$ Recall that $\widetilde \omega = \Omega^k_L$ on $L$ and, therefore $\Omega_t = \Omega^k_L$ on $L$. Since this form is nondegenerate, by reducing the neighborhoods further, we can assume that $\Omega_t$ is nondegenerate on $V$, for every $t \in [0, 1]$. Notice that $\iota_Y \left( \widetilde \theta - \Theta^k_L \right) = 0$, for any vector field $Y$ that takes values in $W^k_L,$ and that $$\iota_{E_1 \wedge \cdots \wedge E_r} \widetilde \theta = \iota_{E_1 \wedge \cdots \wedge E_r} \Theta^k_L = 0,$$ for vector fields $E_1, \dots, E_r$ such that $\pi(E_i)$ takes values in $\mathcal{E} \subset L$ (where $\pi: \bigwedge  ^k_r L \rightarrow L$ is the canonical projection). These last two properties, together with \cref{lemma1}, imply that there exists an unique time-dependent vector field $X_t$ with values in $W_L^k$ satisfying $$\iota_{X_t} \Omega_t =  \widetilde \theta - \Theta^k_L.$$ Furthermore, since $\widetilde \theta = \Theta ^k_L = 0$ on $L$, $X_t = 0$ on $L$, and its flow is globally defined on $L$. It follows that we can assume that $\phi_t$ (the flow of $X_t$) is defined on $V$ for $0 \leq t \leq 1$ by reducing the neighborhoods further. Finally, for $t = 1$, this flow satisfies $$\phi_1 ^\ast \widetilde \omega = \Omega$$ and preserves fibers, because $X_t$ takes values in $W_L^k.$ Defining $$\psi := (\phi_1)^{-1} \circ \phi,$$ we get the desired multisymplectomorphism.
\end{proof}

We can use \cref{LagrangianFormMultisymplectic} to give a local form for vertical $k$-coisotropic submanifolds $N \hookrightarrow M$ of a multisymplectic manifold of type $(k,r)$, where vertical means that $$W \big |_N \subseteq TN.$$ 

\begin{theorem}[Local form of $k$-coisotropic submanifolds relative to Lagrangian submanifolds]\label{localformCoisotropicLagrangian}Let $(M, \omega, W, \mathcal{E})$ be a multisymplectic manifold of type $(k, r)$, $i: N \hookrightarrow M$ be a $k$-coisotropic submanifold satisfying $$W|_N \subseteq TN,$$ and $L \hookrightarrow M$ be a $k$-Lagrangian submanifold complementary to $W$, that is, such that $$W |_L \oplus TL = TM |_L.$$ Then there exists a neighborhood $U$ of $L$ in $M$, a submanifold $Q \hookrightarrow L$, a neighborhood $V$ of $L$ in $\bigwedge^k_r L$, and a multisymplectomorphism $$\phi: U \rightarrow V$$ satisfying
\begin{itemize}
    \item[$i)$] $\phi$ is the identity on $L$, identified as the zero section in $\bigwedge^k_r L$;
    \item[$ii)$] $\phi(N \cap U) = \bigwedge^k_r L \big |_Q \cap V.$
\end{itemize}
\end{theorem}
\begin{proof} Let $U$, $V$, and $\phi$ be the neighborhoods and multisymplectomorphism from \cref{normalformlagrangian} and define $$Q := L \cap N.$$ We claim that, for $U,V$ small enough, $$\phi(N \cap U) = \bigwedge^k_r L \big |_Q \cap V.$$ First recall that we have $$\phi_\ast W = W_L,$$ where $W_L$ is the canonical $1$-Lagrangian distribution on $\bigwedge^k_rL.$ Let $x \in L \cap N$ and $F_x$ be the leaf of $W$ through $x$. It is clear that $F_x \subseteq N$, and that, reducing $U$ and $V$ if necessary, $$\phi(F_x \cap U) =\bigwedge^k_r T^\ast_x L \cap V,$$ since diffeomorphisms that preserve distributions preseve their leaves (when the distributions are integrable). Again, reducing $U$ and $V$ further, we may also assume that for every point $y \in N \cap U$ there is a point $x \in L \cap N$ such that the leaf of $W$ that contains $x$, $F_x$, also contains $y$, that is, we may assume that $$N \cap U = \bigcup_{x \in L \cap N} F_x \cap U.$$ Therefore, $$\phi(N \cap U) = \bigcup_{x \in L \cap N} \phi(F_x \cap U) = \bigcup_{x \in Q} \bigwedge^k_r T^\ast_x L \cap V = \bigwedge^k_rL \big |_Q \cap V,$$ proving the result.
\end{proof} 

\begin{theorem}\label{localformcoisotropic} Let $(M, \omega, W, \mathcal{E})$ be a multisymplectic manifold of type $(k,r)$, and $N \hookrightarrow M$ be a $k$-coisotropic submanifold satisfying $$W|_N \subseteq TN.$$ Then, given any point $x \in N$, there exists a neighborhood $U$ of $x$ in $M$, a manifold $L$, a submanifold $Q \hookrightarrow L$, a neighborhood $V$ of $L$ in $\bigwedge^k_r L$ and a multisymplectomorphism $$\phi: U \rightarrow V$$ such that
\begin{itemize}
    \item[$i)$] $\phi$ is the identity on $L$, idetified as the zero section in $\bigwedge^k_r L$;
    \item[$ii)$] $\phi(N \cap U) = \bigwedge^k_r L \big |_Q \cap V.$
\end{itemize}
\end{theorem}
\begin{proof} Using \cref{thm:darbouxmultisymplectic}, we can build a $k$-Lagrangian submanifold $L$ through any given point $x \in N.$ Now the result follows using \cref{localformCoisotropicLagrangian}.
\end{proof}